The ICARUS T600 LArTPC detector successfully ran for three years at the underground LNGS laboratories, providing a first sensitive search for LSND-like anomalous electron neutrino appearance in the CNGS beam. After a significant overhauling at CERN, the T600 detector has been placed in its experimental hall at Fermilab, fully commissioned, and the first events observed with full detector readout. Regular data-taking began in May 2021 with neutrinos from the Booster Neutrino Beam (BNB) and neutrinos six degrees off-axis from the Neutrinos at the Main Injector (NuMI). Modern developments in machine learning have allowed for the development of an end-to-end machine-learning-based event reconstruction for ICARUS data. This reconstruction folds in 3D voxel-level feature extraction using sparse convolutional neural networks and particle clustering using graph neural networks to produce outputs suitable for physics analyses. The analysis presented here demonstrates a high-purity and high-efficiency selection of muon neutrino interactions in the BNB suitable for the physics goals of the ICARUS experiment and the Short-Baseline Neutrino Program.