The analysis presented here has investigated the performance of the end-to-end machine learning reconstruction chain for the ICARUS T600 detector in three different muon neutrino signal channels: a 1$\mu$1$p$ channel, a 1$\mu$N$p$ channel, and an inclusive $\nu_\mu$ CC channel. The first two exclusive channels are most relevant for the near-term single-detector analysis goals of ICARUS, while the inclusive channel is a more general test of the performance of the machine learning chain and a target for the SBN-wide joint oscillation analysis envisioned in the SBN proposal \cite{Acciarri2015}. 

The performance of the selections in the three signal channels is quantified in terms of purity and efficiency. This analysis has demonstrated that the selections in the two exclusive channels have achieved a purity of at least 80\% and an efficiency of at least 70\% with all relevant backgrounds included. The inclusive channel has achieved a purity of 90\% and an efficiency of 83\%. In all cases, cosmic backgrounds are subdominant to mis-reconstructed neutrino backgrounds and the selections have achieved exceptional purity and efficiency. Given the 80\% efficiency for automated reconstruction outlined in the SBN proposal, this analysis has demonstrated that ICARUS meets one of its core deliverables for the joint SBN analysis. Although the full joint SBN analysis includes both contained and exiting events, this analysis can be easily extended to exiting events and is a crucial step towards the ultimate goal. Of particular note on this topic is the fact that this reconstruction has been implemented for SBND as well, meaning that this reconstruction chain has no technical hurdles for use in the joint SBN analysis.

Detector systematics have been characterized in this analysis by producing simulation samples that implement single variations in the underlying parameters controlling the detector simulation. This analysis has shown that the only significant detector systematic among those studied is the systematic associated with the TPC signal response shape on the waveforms. Though this variation is expected to be conservative, more work is required within the ICARUS collaboration to more precisely characterize and mitigate detector systematics in order to reach 2-3\% systematic uncertainty from the detector model assumed in the SBN proposal. The current level of detector systematics is not expected to be a limiting factor in ICARUS-only analyses as the neutrino interaction model dominates the systematic uncertainty in the selection.

Agreement between data and simulation is generally good across variables that probe a wide region of phase space. Two notable exceptions are the overall higher number of candidate interactions selected in data with respect to the simulation and the disagreement observed in the muon PID score variable. The former disagreement is not well-correlated with any particular distribution and is currently thought to be a result of the chosen central value parameters used in the neutrino interaction model. The ICARUS collaboration is exploring a tune of the central value parameters to better match external data. The latter disagreement may be due to the use of plane-averaged charge rather than collection-only charge, thus causing a bias in data with respect to simulation. Moreover, the electron lifetime in the Monte Carlo simulation is different from the value observed in data and may contribute to this discrepancy. This is actively being investigated at the time of writing. Overall, there are no substantial disagreements between data and simulation that would suggest an inefficiency in the selection that is correlated with a particular variable.

In conclusion, this analysis has demonstrated that the ICARUS T600 detector is capable of achieving high purity and efficiency in the selection of contained muon neutrino interactions using an end-to-end machine learning reconstruction chain. This analysis meets the deliverables outlined in the SBN proposal for the BNB muon neutrino disappearance channel of the joint oscillation analysis, and can be readily extended to the NuMI beam at ICARUS and to the SBND detector. The inclusion of un-contained muon neutrino interactions in the selection to increase statistics is a natural extension of this analysis and is a target for future work. Work is also ongoing to develop analyses with this reconstruction chain targeting electron neutrinos from the BNB and the NuMI beam. This analysis represents a critical step forward in achieving the ultimate goal of the SBN Program: a joint oscillation analysis that combines the precise reconstruction and selection of muon neutrino and electron neutrino interactions from the BNB and NuMI beams at ICARUS and the BNB at SBND. The ICARUS collaboration is well-positioned to contribute to this goal with the work presented in this technical note.